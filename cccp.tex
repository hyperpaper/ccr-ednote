\documentclass[sigconf]{acmart}

% Disable / remove copyright boxes
\setcopyright{none}
\settopmatter{printacmref=false}
\renewcommand\footnotetextcopyrightpermission[1]{}

% Increase margin between text and footer
\setlength{\footskip}{20pt}

% Add CCR footer
\usepackage{fancyhdr}
\fancypagestyle{plain}{%
   \fancyhf{} %
   \fancyfoot[L]{ACM SIGCOMM Computer Communication Review}%
   \fancyfoot[R]{Volume 49 Issue 2, April 2019}%
}
\pagestyle{plain}

% Add CCR footer on first page
\fancypagestyle{firstpagestyle}{%
   \fancyhf{} %
   \fancyfoot[L]{ACM SIGCOMM Computer Communication Review}%
   \fancyfoot[R]{Volume 49 Issue 2, April 2019}%
}

% Add editorial note
\begin{teaserfigure}
	\parbox{\textwidth}{\centering\normalsize
		This article is an editorial note submitted to CCR. It has NOT been peer reviewed.\\
		The authors take full responsibility for this article's
		technical content. Comments can be posted through CCR Online.
	}
	\vspace{10pt}
\end{teaserfigure}

\usepackage{balance}

\begin{document}

\title{Open Collaborative Hyperpapers: A Call to Action}

\author{Alberto Dainotti}
\affiliation{
	\institution{CAIDA, UC San Diego, USA}
}
\email{alberto@caida.org}

\author{Ralph Holz}
\affiliation{
	\institution{University of Sydney, Australia}
}
\email{ralph.holz@sydney.edu.au}

\author{Mirja K\"uhlewind}
\affiliation{
	\institution{ETH Z\"urich, Switzerland}
}
\email{mirja.kuehlewind@tik.ee.ethz.ch}

\author{Andra Lutu}
\affiliation{
	\institution{Telef\'onica Research, Spain}
}
\email{andra.lutu@telefonica.com}

\author{Joel Sommers}
\affiliation{
	\institution{Colgate University, USA}
}
\email{jsommers@colgate.edu}

\author{Brian Trammell}
\affiliation{
	\institution{ETH Z\"urich, Switzerland}
}
\email{brian@trammell.ch}


\begin{abstract}
	Drawing on discussions at various venues, we envision a publishing ecosystem
	for Internet science, supporting publications that are self-contained,
	interactive, multi-level, open, and collaborative. These publications, which
	we dub \emph{hyperpapers}, not only address issues with reproducibility and
	verifiability of research in Internet science and measurement, but have the
	potential to increase the impact of our work and change how collaborations
	work in the field. This note announces initial experiments with Internet
	measurement hyperpapers with the help of common, tested technologies in data
	science and software development, and is a call to action to others to come
	build out this vision with us.
\end{abstract}

\maketitle

\section{Motivation}\label{sec:intro}

Scientific papers were born as a means to share novel scientific knowledge.
However, over time publications have also become the main metric for career
advancement. This shift has influenced the whole publishing process, from the
generation of ideas, data and results to how they are shared. If we step back
and look at the currently established process for scientific paper authoring and
publishing, including conventions and formats, it is clear there is room for
optimization for the good of science and education (e.g.,  have we struck the
right balance between “secrecy” and openness? Are there opportunities from
recent technologies and collaborative practices that we can leverage?)

Discussion at various venues, including the CAIDA AIMS workshop in March 2018
and the seminar  “Encouraging Reproducibility in Scientific Research of the
Internet” at Schloss Dagstuhl in October 2018, identified issues with the
publishing ecosystem through which Internet measurement studies are disseminated
that have an impact on reproducibility and verifiability of Internet science.

The current publication ecosystem discourages incremental work. Together with an
academic emphasis on novelty (whether that novelty is of improved utility or
not), the difficulty of building on existing studies and experiments means few
researchers bother. Some venues in our field are starting to experiment with
artefacts such as code and data being submitted together with papers. While this
is far better than the common status quo of boutique code running on secret
data, these efforts are recent, and are more suited to address archival
requirements than those of repeatability and verifiability. 

In addition, tooling support is lacking to help reviewers to efficiently verify
these artefacts as part of the paper review process, for example through
repetition of experiments or analysis. The difficulty this causes will make it
difficult to make partial verification of results during review the norm.

A culture of secrecy, both due to conditions of access on proprietary data
sources as well as to the tradition of establishing academic priority, adds
further barriers to scientific inquiry. Dead-ends and negative results stay
secret within the groups that find them. Researchers new to the field, and those
without connections to established cliques of collaborators, can find it hard to
get started in impactful Internet measurement. 

It is not the goal of this note to advocate for the immediate death of the
current publication process. Indeed, this process performs a valuable gatekeeper
function, both by providing incentives for authors to submit papers for review
and for reviewers to review them, as well as for bringing work from smaller
communities (in our case, Internet measurement) to larger audiences. The spirit
of our proposal is instead to suggest ways in which we, as a community, can
address the issues above while retaining positive aspects of the current
publication process.

\section{A Vision for the Future}\label{sec:vision}

We propose to leverage recent developments in platforms and tools for data
science and scientific collaboration to build an experimental publishing
ecosystem for Internet measurements based on hyperpapers. The technical details
of this experiment are left to future work, but the outlines of the properties
of these papers already seem clear:

\textbf{Hyperpapers are self-contained and interactive.} Ideally, a full
hyperpaper contains all the data from which results, plots, and conclusions in
the paper are drawn, as well as source code implementing the analytic tasks
distilling those results from the raw source data. The paper is interactive,
allowing both changes to the raw source data and to the analysis code to be
reflected in the analytic products (tables, charts, etc) in the paper,
supporting easy exploratory analysis of a dataset and/or measurement analysis
methodology as well as incremental changes to published work. In any case, the
hyperpaper includes one or more rendered versions, both for compatibility with
existing publishing channels (e.g., PDF) and to minimize the requirements for
the (perhaps vast majority) of readers who will not want to use the paper’s
interactive features.

The raw source data may be included with the hyperpaper either by value or by
reference, and the analysis code can be configured to run either on a local
machine or on some specified remote infrastructure. Hyperpapers can also require
credentials to enable their interactivity, in order to support arrangements
where raw and/or intermediate data is subject to access control.

\textbf{Hyperpapers are multilevel.} The initial view a reader will have of a
full hyperpaper includes the typical prose of a paper (an abstract,
introduction, explanation of the question answered, description of the
methodology, results, graphs, and so on). Analysis products, such as charts and
tables, can be expanded to show how they were derived.

However, the paper can also be expanded in other ways. A section of prose may be
linked to an alternate view, information for an alternate audience, related
content, or a drill down on some interesting set of a result. For example, the
paper could contain an executive summary describing the utility of its insights
for the general public, network operators, regulators, and so on; a methodology
could be expanded to tell an expanded narrative of roads not taken, and why; or
an introduction could be expanded with introductory information on a measured
protocol not necessary for practitioners in the field but useful for students
just learning it; and so on.  These sorts of expansions would replace the
creation of multiple papers on a subject to multiple venues,
e.g.~\cite{Dhamdhere18} and \cite{Clark18}.

Taken to its logical conclusion, multilevel hyperpapers allow a full explanation
and report on a research project’s activity and the evolution of the chain of
hypotheses, addressing the death of negative results, while maintaining
narrative coherence and conciseness in the paper’s main “trunk”. While this
throws up new questions, e.g., for page counts and the review process, this
could be addressed by considering all extensions beyond this trunk as appendices
for review purposes.

\textbf{Hyperpapers are open and collaborative.}  The self-contained,
interactive, and multilevel nature of hyperpapers enables — indeed, may require
— entirely new ways of working together as researchers. Starting incremental
work on a paper, or beginning to verify and reproduce it, becomes a simple
matter of forking it, given permission to access the data. To the extent that
the hyperpaper infrastructure is integrated with a collaboration platform,
papers can become living projects, with researchers performing associated work,
interdisciplinary co-authors writing expansions for specific audiences beyond
the usual networking conference crowd, and even reviewers becoming part of a
long-running, data-centered conversation about a particular hypothesis of how
the Internet is.

\section{How do we get there?}\label{sec:how}

As in all things scientific, by standing on the shoulders of giants. While this
vision may seem like science fiction, akin to mid-century utopian writing about
hypertext before the rise of the Web, we submit that enough of the underlying
infrastructure behind this vision exists that rudimentary experimental
hyperpapers for Internet measurements can be written today.

The perennial problem of setting up environments for data analysis without
needing to replicate a full toolchain with dependencies from scratch is largely
solved today by virtualization and containerization tools such as Vagrant and
Docker. Problems of scale are addressed by the easy (if sometimes costly)
widespread availability of cloud infrastructure from multiple providers.
Integration of data analytics with authoring environment interleaving text and
interactive visualizations is supported by data analysis notebooks such as
JupyterLab and Apache Zeppelin. GitHub has emerged as the de facto standard for
integrating version control of digital artefacts with a collaboration
environment, and its model of working is suited to open collaborative papers as
we envision them, which have a fair amount in common with the long-running open
source projects GitHub was originally built to support. Of course, all of these
technologies are supported by the web platform, decades of continuous investment
in which has brought us to a world where almost all of the work of research can
be done in a standard browser.

We have identified two main gaps in technical infrastructure necessary for a
full initial realization of this vision:

\begin{itemize}
	
	\item First, while some research studies can be done with data or models
	that can easily be stored in an ad-hoc format within the hyperpaper itself,
	large-scale Internet measurement studies need access to large data sets
	mediated through some interface. This exists for certain data sources (e.g.,
	the RIPE Atlas API), but a full realization would require the creation and
	standardization of interfaces for retrieval of data and metadata for each
	broad type of measurement activity.

	\item Second, the distribution of rendered versions of papers is currently
	possible for scientific notebook environments, but these render to a webpage
	that is not necessarily optimized for accessibility. Tooling to render a
	view of hyperpaper as a PDF according to the visual style for a given venue,
	for example, is necessary to support the full multi-rendering functionality
	of the vision above. We consider this a simple matter of engineering,
	though.

\end{itemize}

Addressing these gaps will take time; in the meantime, the authors are currently
working to build initial hyperpaper versions of some of their previous and
current work, one of which~\cite{Trammell17} has already been built using some
of the technologies we identify as appropriate for hyperpapers. We welcome
community collaboration as we develop it into an architecture document for an
initial realization of a hyperpaper platform; see our GitHub organization at
\url{https://github.com/hyperpaper}.

\begin{acks}
	This editorial is largely based on discussions at CAIDA's AIMS workshop in
	March 2018 and Dagstuhl Seminar 18412 on \emph{Encouraging Reproducibility
	in Scientific Research of the Internet}; we thank the participants for the
	discussions leading to this work.
\end{acks}

{ \balance
{
	\bibliographystyle{ACM-Reference-Format}
	\bibliography{cccp}
}
}

\end{document}
